\chapter{Perfect Secrecy}

A ciphertext is perfectly secret if it reveals nothing about the plaintext.

Let $M$ be the space of messages (plaintext), $C$ be the space of ciphertexts and $k$ a key.\\
$\forall m \in M$, $\forall c \in C$, let $C = Enc(k,M)$
\[ Pr[M = m] = Pr[M=m \vert C=c] \]

Some equivalent definitions of perfect secrecy:
\begin{itemize}
    \item[-] $Pr[M = m] = Pr[M=m \vert C=c]$
    \item[-] $M$ and $C$ are independent
    \item[-] $\forall m,m' \in \mathcal{M}$, $\forall c \in C$ $Pr[Enc(k,m)=c] = Pr[Enc(k,m')=c]$  
\end{itemize}


\section{One Time Pad - OTP}
Let $k,m,c \in \bin^\lambda$

$Enc(k,m) = c = k \oplus m$\\
$Dec(k,c) = c \oplus k = (k \oplus m) \oplus k = m$

\begin{theorem}
    Above OTP is perfectly secret.
\end{theorem}

\begin{proof}
$ $\newline
$Pr[Enc(k,m)=c] = Pr[k \oplus m = c] = Pr[k = c \oplus m] = 2^{-\lambda}$\\
$Pr[Enc(k,m')=c] = Pr[k \oplus m' = c] = Pr[k = c \oplus m'] = 2^{-\lambda}$
\end{proof}

Limitations:
\begin{itemize}
    \item[-] $\cardinality{k} = \cardinality{m}$
    \item[-] One time 
\end{itemize}